%!TEX TS-program = xelatex
\documentclass[]{friggeri-cv}

\begin{document}
\header{Danbee}{Kim}
       {Field Neuroscientist | Teacher | Author}


% In the aside, each new line forces a line break
\begin{aside}
  \section{contact}
    \href{mailto:danbee@alum.mit.edu}{danbee@alum.mit.edu}
    \href{www.danbeekim.org}{www.danbeekim.org}
  \section{languages}
    English (fluent),
    Korean (intermediate),
    Portuguese (beginner)
  \section{computer skills}
    Microsoft Office Suite, 
	Illustrator, Photoshop, 
	Premier Pro, Bonsai-Rx, 
	Github, \LaTeX, Python
  \section{fabrication tools}
    Epilog Helix Laser Cutter, sewing machine (standard and overlocking)
  \section{movement arts}
    capoeira, fire spinning,
    musical theater
  \section{instruments}
    bass guitar, ukelele,
    violin, vocals
  \section{references}
    available upon request        
\end{aside}

\section{mission}
%\textbf{Field Neuroscience}\\
To create interactive, empathic experiences that empower the study of nervous systems in natural settings. \\
%\textbf{Storytelling for Science Literacy}\\
%To create interactive, empathic experiences that nuture a global scientific literacy. 

\section{education}

\begin{entrylist}
  \entry
    {2013-2020}
    {Doctor of Philosophy in Neuroscience}
    {Champalimaud Centre for the Unknown}
    {International Neuroscience Doctoral Programme, \href{http://neuro.fchampalimaud.org/en/research/investigators/research-groups/group/Kampff/}{Intelligent Systems Lab} \\ Thesis title: \href{https://www.dropbox.com/s/iqyfmj1rmcn083z/DanbeeKim_INDP2013_PhDThesis.pdf?dl=0}{\emph{On the aims and methods of Field Neuroscience: Non-invasive techniques for studying nervous systems in natural settings}}}
    %Online open lab notebook: \href{http://www.danbeekim.org/open-lab-notebook/}{www.danbeekim.org/open-lab-notebook/}}
  \entry
    {2005–2009}
    {Scient\ae Bacculaureus}
    {Massachusetts Institute of Technology}
	{\emph{Brain and Cognitive Sciences}}
\end{entrylist}

\section{experience}

\begin{entrylist}
  \entry
    {since 2021}
    {\href{https://neurogears.org/}{NeuroGEARS, Ltd}}
    {London, UK}
    {\emph{Research Scientist, Outreach}}
  \entry
    {since 2021}
	{The Bartlett School of Architecture, UCL}
	{London, UK}
	{\emph{Tutor, MArch Design for Performance and Interaction}}
  \entry
	{since 2021}
	{\href{https://terrascope.mit.edu/}{Terrascope}, MIT}
	{Cambridge, Massachusetts, USA}
	{\emph{Alumni Mentor}}
  \entry
    {since 2010}
    {\href{http://www.appalachianinstitute.org/}{Appalachian Institute for Creative Learning}}
    {Mars Hill, North Carolina, USA}
    {\emph{Teacher, Staff}}
\end{entrylist}

\section{publications and exhibits}

Danbee Kim, Kendra Buresch, Roger Hanlon, Adam R. Kampff. ``An experimental method for evoking and characterizing dynamic color patterning of cuttlefish during prey capture''. \emph{In prep}.

\vspace{1.2mm}
Danbee Kim. \emph{The First VIRS}. London; 2020. 

\vspace{1.2mm}
Danbee Kim, Adam R. Kampff. \href{https://sci-hub.do/https://onlinelibrary.wiley.com/doi/abs/10.1002/ad.2637}{``Neuroscience Does Design: What the Brain's Architecture Can Teach Architects''}. \emph{Architectural Design}, 90.6 (2020): 94-99. DOI: 10.1002/ad.2637.

\vspace{1.2mm}
Andr\'{e} Marques-Smith, Joana P. Neto, Gon\c{c}alo Lopes, Joana Nogueira, Lorenza Calcaterra, Jo\~{a}o Fraz\~{a}o, Danbee Kim, Matthew G. Phillips, George Dimitriadis, Adam R. Kampff. \href{https://www.biorxiv.org/content/10.1101/370080v1}{``Recording from the same neuron with high-density CMOS probes and patch-clamp: a ground-truth dataset and an experiment in collaboration''}. \emph{Bioarxiv}, 2018. DOI: 10.1101/370080.

\vspace{1.2mm}
Danbee Kim. \href{https://massivesci.com/articles/frankenstein-kim-animal-testing/}{“Why I refuse to do animal testing in my science career”}. \emph{\href{https://massivesci.com}{Massive Science}}. June 18, 2018.

\vspace{1.2mm}
Dar\'{i}o R. Qui\~{n}ones, Gon\c{c}alo Lopes, Danbee Kim, C\'{e}dric Honnet, David Moratal, Adam Kampff. \href{https://www.researchgate.net/publication/322842913_HIVE_Tracker_a_tiny_low-cost_and_scalable_device_for_sub-millimetric_3D_positioning}{"HIVE Tracker: a tiny, low-cost, and scalable device for sub-millimetric 3D positioning"}. \href{http://www.sigah.org/AH2018/}{\emph{Augmented Human}, 9 (2018)}. DOI: 10.1145/3174910.3174935.

\vspace{1.2mm}
Gon\c{c}alo Lopes, Danbee Kim. \href{https://massivesci.com/articles/neuroscience-can-learn-from-theater/}{"How theater, start-up culture, and business history helped us become better neuroscientists"}. \emph{\href{https://massivesci.com}{Massive Science}}. Oct 16, 2017.

\vspace{1.2mm}
Danbee Kim, Gon\c{c}alo Lopes. \href{https://massivesci.com/articles/neuroscience-behavior-vs-technology/}{"Does modern neuroscience really help us understand behavior?"} \emph{\href{https://massivesci.com}{Massive Science}}. Oct 3, 2017. 

\vspace{1.2mm}
\emph{Surprising Minds}. Interactive installation and crowd human behaviour experiment. Installed 4 July 2017 at Sea Life Brighton, Brighton, UK. Danbee Kim, Kerry Perkins, Clive Ramble, Hazel Garnade, Gon\c{c}alo Lopes, Dar\'{i}o R. Qui\~{n}ones, Reanna Campbell-Russo, Robb Barrett, Martyn Stopps, The EveryMind Team, Adam Kampff.

\newpage
\section{experience continued...}
\begin{entrylist}
%------------------------------------------------
\entry
    {2016-2019}
    {\href{https://www.ucl.ac.uk/swc/}{Sainsbury Wellcome Centre for Neural Circuits and Behaviour}}
    {London, UK}
    {\emph{Visiting Researcher, Intelligent Systems Lab} \\
	\textbf{Awards:}
	\begin{itemize}
		\item SWC Public Engagement Fund: £3,000 (Feb 2019)
		\item UCL Train and Engage program: £1,000 (July 2018)
	\end{itemize}
	\textbf{Events:}
	\begin{itemize}
		\item \href{http://www.everymind.online/DearNeuroscience/TouchProprioception/}{"Dear Neuroscience: Touch and Proprioception"}, Organizer and Speaker (24 May 2019)
		\item 2017 Systems Seminars Annual Symposium: Cross-Species Conversations, Organizing team (21 Sept 2018)
		\item \href{https://www.sainsburywellcome.org/web/public-engagement/orchestrating-brain}{"Orchestrating the Brain: A collaborative workshop of musicians and neuroscientists"}, \\Organizer and Speaker (31 May 2018)
	\end{itemize}
	\textbf{Societies and Working Groups:}
	\begin{itemize}
		\item SWC-Gatsby PhD Society, Social Chair (May 2018 -- May 2019)
		\item SWC Staff Student Consultative Committee, Student representative (Oct 2017 -- Oct 2018)
		\item SWC NC3Rs working group, Member (July 2017 -- July 2019)
		\item SWC Public Engagement Network, Member (Jan 2016 -- July 2019)
	\end{itemize}
	}
%------------------------------------------------
  \entry
	{2005--2019}
	{\href{http://web.mit.edu/mtg/www/}{MIT Musical Theater Guild}}
	{Cambridge, Massachusetts, USA}
	{\emph{Member}
	\begin{itemize}
		\item Corresponding Secretary (2013--2019)
		\item Costume Shop Manager (2007--2009)
	\end{itemize}
	\textbf{Shows:}\\
	\emph{9 to 5}, 2016: vocal director\\
	\emph{Spring Awakening}, 2015: pit orchestra (violin and guitar)\\
	\emph{Legally Blonde}, 2014: co-choreographer\\
	\emph{Sweeney Todd}, 2014: vocal director, pit orchestra (violin)\\
	\emph{Reefer Madness}, 2012: choreographer\\
	\emph{Urinetown}, 2012: Ma Strong, ensemble\\
	\emph{Hack, Punt, Tool}, 2012: co-writer, choreographer\\
	\emph{Children of Eden}, 2011: vocal director, Snake\\
	\emph{Assassins}, 2011: Charles Guiteau, co-props\\
	\emph{25th Annual Putnam County Spelling Bee}, 2011: vocal director, pit orchestra (violin)\\
	\emph{Jekyll and Hyde}, 2011: co-director, choreographer\\
	%\end{itemize}
	%}
	%
	%\end{entrylist}
	%
	%\section{experience continued...}
	%\begin{entrylist}
	%
	%\partentry
	%{\small 
	%\begin{itemize}
	\emph{Evil Dead}, 2010: Annie, master seamstress\\
	\emph{Little Shop of Horrors}, 2010: assistant choreographer\\
	\emph{Side Show}, 2009: choreographer\\
	\emph{Bare}, 2009: Kyra; program designer, master seamstress\\
	\emph{The Mystery of Edwin Drood}, 2009: Angela Prysock/Princess Puffer; costume designer\\
	\emph{Wild Party}, 2008: Kate\\
	\emph{Pippin}, 2007: Bertha, Manson Trio; co-costume designer\\
	\emph{Cabaret}, 2007: costume designer\\
	\emph{Reefer Madness}, 2007: Mae; props designer\\
	\emph{Children of Eden}, 2006: Eve; costume designer\\
	\emph{Crazy For You}, 2006: Everett Baker\\
	\emph{Chicago}, 2006: director\\
	\emph{Urinetown}, 2006: Hot Blades Harry\\
	\emph{Star Wars: The Musical}, 2005: Bail Organa, Lobot, ensemble
	}
%------------------------------------------------
  \entry
	{2011--2013}
	{\href{http://hackpunttool.com/}{Hack, Punt, Tool}}
	{Cambridge, Massachusetts, USA}
	{\emph{Co-writer}
	\begin{itemize}
		\item Co-wrote \href{http://hackpunttool.files.wordpress.com/2012/03/hptfinalscript.pdf}{script} and contributed to \href{http://hackpunttool.files.wordpress.com/2012/03/hpt-pc-score-2-16-12.pdf}{music} to create an original show about hacking culture and life at MIT
		\item Collaborated with MIT administration, teachers, and students to create a work that has a significant positive impact on the MIT community
		\item Produced by the \href{http://web.mit.edu/mtg/www/2012/IAP/ProdStaff.html}{MIT Musical Theater Guild during IAP 2012}
		\item Writing and music teams recorded and mastered an \href{http://hackpunttool.bandcamp.com/}{original cast recording, released in Sept 2012}
		\item Released a \href{https://www.youtube.com/playlist?list=PLEUCiGVkvGkd9ZECCR2aLecWmMHvccJtI}{subtitled video recording of the MIT production on YouTube in Sept 2013}
	\end{itemize}
	}
%----------------------------------
\end{entrylist}

\newpage
\section{experience continued...}
\begin{entrylist}
%------------------------------------------------
  \entry
	{2011--2012}
	{\href{http://www.mos.org/}{Museum of Science}}
	{Boston, Massachusetts, USA}
	{\emph{Education Associate, Current Science \& Technology}
	\begin{itemize}
		\item developed, and performed 20-minute presentations on science and technology topics
		\item contacted and coordinated guest presenters
		\item organized logistics for Museum events
	\end{itemize}
	}
%------------------------------------------------
  \entry
	{2009--2011}
	{\href{http://www.bidmc.org/}{Harvard Medical School, Beth Israel Deaconess Medical Center}}
	{Boston, Massachusetts, USA}
	{\emph{EEG Lab Technician, Research Assistant}
	\begin{itemize}
		\item organized and managed EEG lab, Psychiatry Suite of BIDMC West Campus
		\item designed and implemented EEG protocols written in Superlab and Presentation software 
		\item manage subject recruitment, coordination with clinical assessments, and payment
	\end{itemize}
	}
%------------------------------------------------
  \entry
	{2009--2010}
	{\href{https://twitter.com/roflcon}{ROFLCon}}
	{Boston, Massachusetts, USA}
	{\emph{Staff}
	\begin{itemize}
		\item coordinated guest travel/lodging and event volunteers
		\item organized event AV logistics
	\end{itemize}
	}
%------------------------------------------------
  \entry
	{2008--2009}
	{\href{http://bcs.mit.edu/}{Department of Brain and Cognitive Sciences, MIT}}
	{Cambridge, Massachusetts, USA}
	{\emph{Undergraduate Researcher} \\
	\textbf{How Expectations Can Change Perception} \\
	Higher-Level Cognition Lab: Talia Konkle, Steven Piantadosi, Rebecca Saxe
	\begin{itemize}
		\item studied the effect of prior expectations on the perception of incongruent stimuli
		\item designed and coded experimental tasks in Matlab; analyzed data in R
	\end{itemize}
	\textbf{Observing Causal Laws by Tracking Eye Movements} \\
	Early Childhood Cognition Lab: Elizabeth Bonawitz, Laura Schulz
	\begin{itemize}
		\item studied how young children learn to make predictions based on patterns
		\item tracked eye movements using Tobii Eyetracker software; analyzed data using Matlab
		\item studies were conducted at the Learning Lab at the Children's Museum of Boston
	\end{itemize}
	}
%------------------------------------------------
  \entry
	{2005--2008}
	{\href{http://web.mit.edu/firstyear/}{Freshmen Pre-Orientation Programs, MIT}}
	{Cambridge, Massachusetts, USA}
	{\emph{Film Counselor for Freshmen Arts Pre-Orientation (FAP)}
	\begin{itemize}
		\item participated in FAP 2005; film counselor for FAP 2006, 2007, and 2008
		\item organized projects and activities for the week-long program
		\item co-wrote, filmed, and edited counselor introduction videos and a yearly FAP video
	\end{itemize}
	}
%------------------------------------------------
  \entry
	{2006--2008}
	{\href{http://web.mit.edu/senior-house/www/steerroast.html}{Senior Haus Annual Steer Roast}}
	{Cambridge, Massachusetts, USA}
	{\emph{Food Veep}
	\begin{itemize}
		\item organized an outdoor feast for approximately 400 people
		\item worked with fellow veeps and MIT staff on event registration, logistics, funding, and safety
		\item coordinated shopping trips and the borrowed use of an industrial kitchen
		\item trained an apprentice and contributed to a Food Veep Bible
	\end{itemize}
	}
%------------------------------------------------
  \entry
	{2005--2008}
	{\href{http://web.mit.edu/terrascope/www/}{Terrascope, MIT}}
	{Cambridge, Massachusetts, USA}
	{\emph{Undergraduate Teaching Fellow, Kitchen and Snacks Coordinator} \\
	Terrascope is a year-long freshmen seminar that examines complex real-world problems, presents potential solutions to a visiting board of experts at the end of fall term, then creates a museum exhibit during spring term. 
	\begin{itemize}
		\item participated as a freshman in \href{http://web.mit.edu/12.000/www/m2009/finalwebsite/}{Mission 2009: The Tsunami Threat to the Pacific}
		\item mentored as an Undergraduate Teaching Fellow in \href{http://web.mit.edu/12.000/www/m2011/finalwebsite/}{Mission 2011: Saving the Oceans}
		\item worked within a budget to stock and maintain the Terrascope kitchen
	\end{itemize}
	}
%------------------------------------------------
  \entry
	{2007}
	{\href{http://edgerton.mit.edu/outreach}{Edgerton Center Outreach Program, MIT}}
	{Cambridge, Massachusetts, USA}
	{\emph{Teaching Assistant} 
	\begin{itemize}
		\item taught grade-school children topics in science and technology via hands-on classroom projects, including motorized Lego cars, rudimentary circuits, high speed photography, and basic chemistry
	\end{itemize}
	}
%------------------------------------------------
\end{entrylist}

\end{document}
